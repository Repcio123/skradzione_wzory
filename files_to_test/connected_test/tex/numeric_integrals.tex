\documentclass{article}
\usepackage{amsmath}
\usepackage{graphicx}
\usepackage{float}
\usepackage{geometry}
\usepackage{hyperref}

\title{Numerical Integration Methods: Theory and Applications}
\author{Your Name}
\date{\today}

\begin{document}

\maketitle

\begin{abstract}
Numerical integration, also known as quadrature, is a fundamental technique in mathematics and scientific computing. In this paper, we explore the theory and applications of numerical integration methods, including the trapezoidal rule, Simpson's rule, and Gaussian quadrature. We discuss the accuracy, convergence, and practical implementation of these methods, along with real-world applications in areas such as physics and engineering.
\end{abstract}

\section{Introduction}

Numerical integration, or quadrature, is the process of approximating the definite integral of a function using a finite number of evaluations. It plays a crucial role in mathematics, scientific computing, and various fields of science and engineering. In this paper, we will examine different numerical integration methods, their theoretical foundations, and practical applications.

\section{Trapezoidal Rule}

The trapezoidal rule is one of the simplest numerical integration methods. Given a function $f(x)$ over the interval $[a, b]$, the integral is approximated by dividing the interval into $n$ equal subintervals and using trapezoids to approximate the area under the curve. The formula for the trapezoidal rule is as follows:

\begin{equation}
\int_{a}^{b} f(x) dx \approx \frac{b - a}{2n} \left[f(a) + 2f(x_1) + 2f(x_2) + \ldots + 2f(x_{n-1}) + f(b)\right],
\end{equation}

where $x_i = a + i\Delta x$ and $\Delta x = (b - a)/n$.

\section{Simpson's Rule}

Simpson's rule is a more accurate numerical integration method that approximates the integral using quadratic interpolating polynomials. The formula for Simpson's rule is given by:

\begin{equation}
\int_{a}^{b} f(x) dx \approx \frac{b - a}{3n} \left[f(a) + 4f(x_1) + 2f(x_2) + 4f(x_3) + 2f(x_4) + \ldots + 2f(x_{n-2}) + 4f(x_{n-1}) + f(b)\right].
\end{equation}

Simpson's rule is often more accurate than the trapezoidal rule and requires fewer function evaluations for the same level of accuracy.

\section{Gaussian Quadrature}

Gaussian quadrature is a family of numerical integration methods that aim to find the best set of nodes and weights to approximate integrals. The idea is to choose nodes and weights that make the method as accurate as possible for polynomials of a given degree. There are different sets of nodes and weights for various degrees of accuracy.

\section{Accuracy and Convergence}

The accuracy of a numerical integration method depends on the number of subintervals or nodes used. Both the trapezoidal rule and Simpson's rule have known error estimates, and the error decreases as the number of subintervals or nodes increases. Gaussian quadrature methods achieve high accuracy for a given number of nodes by carefully selecting them.

\section{Practical Applications}

Numerical integration methods are widely used in various fields, including physics, engineering, finance, and computer science. They are essential for solving problems that involve the computation of integrals that cannot be solved analytically. Real-world applications include the computation of areas, volumes, probabilities, and solving differential equations.

\section{Conclusion}

Numerical integration methods are indispensable tools in mathematics and scientific computing. They provide efficient and accurate approximations of integrals and have diverse applications in various scientific and engineering disciplines. Understanding the theory and practical implementation of these methods is crucial for solving real-world problems and advancing scientific knowledge.

\end{document}
