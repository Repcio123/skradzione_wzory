\documentclass{article}
\usepackage{amsmath}

\title{The Physics of Dwarf Wrestling}
\author{Carmen E. Wendell, Brent H. Anders}
\date{April 4, 2015}

\begin{document}

\maketitle

\section{Introduction}

Dwarf wrestling, a captivating form of combat sport, has garnered attention and admiration in recent times. This article explores the world of dwarf wrestling from a mathematical and physical perspective, delving into the intricacies of the sport. We will derive detailed equations to model the dynamics of a wrestling match involving two dwarfs.

\section{Modeling Dwarf Wrestling Dynamics}

To gain a deeper understanding of dwarf wrestling, let's analyze the forces and torques at play during a match. Consider two dwarfs, each with a mass $m$ and a height $h$, engaged in a wrestling match. The gravitational force acting on each dwarf is given by:

\begin{equation}
F_{\text{gravity}} = m \cdot g,
\end{equation}

where $g$ is the acceleration due to gravity.

Now, when the dwarfs grapple, they exert forces and torques on each other. To maintain balance, the sum of forces and torques acting on a dwarf must be zero. We can write these equations of equilibrium for each dwarf:

\begin{equation}
\sum F_x = 0, \quad \sum F_y = 0, \quad \sum F_z = 0,
\end{equation}

\begin{equation}
\sum \tau_x = 0, \quad \sum \tau_y = 0, \quad \sum \tau_z = 0,
\end{equation}

These equations represent the balance of forces and torques in the $x$, $y$, and $z$ dimensions.

The torque applied by one dwarf to another can be calculated as the cross product of the lever arm vector $\mathbf{r}$ and the force vector $\mathbf{F}$:

\begin{equation}
\boldsymbol{\tau} = \mathbf{r} \times \mathbf{F},
\end{equation}

The magnitude of the torque is given by:

\begin{equation}
|\boldsymbol{\tau}| = r \cdot F \cdot \sin\theta,
\end{equation}

where $r$ is the lever arm's length, $F$ is the applied force, and $\theta$ is the angle between $\mathbf{r}$ and $\mathbf{F}$.

In dwarf wrestling, maintaining balance while destabilizing the opponent is a delicate art. Dwarfs exert torques to shift their opponent's center of mass, aiming to unbalance them and gain the upper hand. The strategy involved in choosing the right force magnitude and direction is a complex decision-making process.

\section{Conclusion}

Dwarf wrestling combines the grace of balance, the strength of force, and the strategy of torque in a dynamic and entertaining spectacle. By examining the physics and mathematics of dwarf wrestling, we can appreciate the intricate mechanics behind the sport. This mathematical perspective sheds light on the forces, torques, and equilibrium required for success in dwarf wrestling.

In summary, dwarf wrestling is not only a physically demanding sport but also a mathematically rich field of study. Understanding the principles of equilibrium, forces, and torques involved in the match adds a new dimension to the appreciation of this unique and exciting sport.

\end{document}