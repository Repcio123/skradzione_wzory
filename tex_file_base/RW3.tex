\documentclass{article}
\usepackage{amsmath} % Pakiet do wzorów matematycznych

\title{Mathematical Concepts and Formulas}
\author{Your Name}
\date{\today}

\begin{document}

\maketitle

\section{Introduction}
In this article, we will delve into various mathematical concepts and equations, exploring their significance and applications.

\section{The Quadratic Formula}
One of the most fundamental equations in algebra is the quadratic formula:

\begin{equation}
    x = \frac{-b \pm \sqrt{b^2 - 4ac}}{2a}
\end{equation}

It allows us to find the solutions for $x$ in a quadratic equation of the form $ax^2 + bx + c = 0$.

\section{The Summation Notation}
The summation notation is a concise way to represent a sum of terms in a sequence. It's expressed as:

\begin{equation}
    \sum_{k=1}^n a_k = a_1 + a_2 + \ldots + a_n
\end{equation}

This notation is frequently used in various mathematical and statistical contexts.

\section{The Law of Sines}
In trigonometry, the Law of Sines relates the sides and angles of a triangle. It's given by:

\begin{equation}
    \frac{a}{\sin A} = \frac{b}{\sin B} = \frac{c}{\sin C}
\end{equation}

Where $a$, $b$, and $c$ are the sides of the triangle, and $A$, $B$, and $C$ are the opposite angles.

\section{The Mean Value Theorem}
The Mean Value Theorem is an essential result in calculus. It states that for a function $f(x)$ that is continuous on the closed interval $[a, b]$ and differentiable on the open interval $(a, b)$, there exists a point $c$ in $(a, b)$ where the derivative equals the average rate of change:

\begin{equation}
    f'(c) = \frac{f(b) - f(a)}{b - a}
\end{equation}

\section{Conclusion}
Mathematics is a vast and diverse field with numerous concepts and formulas. The equations we've explored in this article offer just a glimpse into the richness of mathematical thought and its applications in various disciplines.

\end{document}
