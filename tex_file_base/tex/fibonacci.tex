\documentclass[12pt]{article}
\usepackage{amsmath, amssymb, amsthm}

\title{Liczby Fibonacciego: Definicje, Własności i Zastosowania}
\author{Twoje Imię i Nazwisko}
\date{\today}

\begin{document}

\maketitle

\section{Wprowadzenie}
Liczby Fibonacciego to ciąg liczb całkowitych, w którym każda liczba jest sumą dwóch poprzednich. Ciąg ten posiada wiele interesujących właściwości matematycznych i znajduje zastosowanie w różnych dziedzinach. W tym artykule omówimy podstawowe definicje związane z liczbami Fibonacciego oraz przedstawimy kilka kluczowych właściwości i zastosowań.

\section{Definicje Podstawowe}
\begin{definition}
Ciąg liczb Fibonacciego jest określony rekurencyjnie jako:
\[
F(n) = F(n-1) + F(n-2)
\]
z warunkami początkowymi $F(0) = 0$ i $F(1) = 1$.
\end{definition}

\section{Właściwości Liczb Fibonacciego}
\begin{theorem}
Liczby Fibonacciego wykazują złoty podział, gdzie iloraz kolejnych liczb dąży do liczby złotej $\phi = \frac{1 + \sqrt{5}}{2}$:
\[
\lim_{n \to \infty} \frac{F(n+1)}{F(n)} = \phi
\]
\end{theorem}

\begin{theorem}
Suma kwadratów kolejnych liczb Fibonacciego wynosi:
\[
F(0)^2 + F(1)^2 + \ldots + F(n)^2 = F(n) \cdot F(n+1)
\]
\end{theorem}

\section{Zastosowania Liczb Fibonacciego}
\begin{itemize}
  \item \textbf{Analiza kombinatoryczna:} Liczby Fibonacciego są często używane w analizie kombinatorycznej do rozważań nad różnymi kombinacjami.
  \item \textbf{Finanse:} W analizie rynków finansowych, liczby Fibonacciego są wykorzystywane w tzw. retracement, pomagając określić poziomy wsparcia i oporu.
  \item \textbf{Algorytmy:} Liczby Fibonacciego znajdują zastosowanie w algorytmach, takich jak algorytm dynamiczny czy programowanie dynamiczne.
\end{itemize}

\section{Podsumowanie}
Liczby Fibonacciego stanowią fascynujący ciąg liczb o wielu interesujących właściwościach matematycznych. W artykule tym omówiliśmy ich podstawowe definicje, przedstawiliśmy kilka kluczowych właściwości, takie jak złoty podział czy suma kwadratów, oraz zastosowania w różnych dziedzinach matematyki i nauki.

\end{document}
