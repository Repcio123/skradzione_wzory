\documentclass{article}
\usepackage{amsmath}
\usepackage{amsfonts}
\usepackage{amssymb}
\usepackage{geometry}

\geometry{a4paper, margin=1in}

\title{Statistics: Exploring Data and Inference}
\author{Your Name}
\date{\today}

\begin{document}

\maketitle

\section{Introduction}

Statistics is a branch of mathematics that deals with the collection, analysis, interpretation, presentation, and organization of data. It plays a crucial role in making informed decisions and drawing meaningful conclusions from uncertain information.

\section{Descriptive Statistics}

\subsection{Measures of Central Tendency}

\subsubsection{Mean}

The mean, or average, of a set of observations $x_1, x_2, \ldots, x_n$ is given by:

\begin{equation}
    \bar{x} = \frac{1}{n} \sum_{i=1}^{n} x_i
\end{equation}

\subsubsection{Median}

The median is the middle value of a data set when it is ordered. For an odd number of observations, it is the middle value; for an even number, it is the average of the two middle values.

\subsubsection{Mode}

The mode is the most frequently occurring value in a data set.

\subsection{Measures of Dispersion}

\subsubsection{Range}

The range of a data set is the difference between the maximum and minimum values.

\begin{equation}
    \text{Range} = \max(x_i) - \min(x_i)
\end{equation}

\subsubsection{Variance}

The variance of a data set is the average of the squared differences from the mean:

\begin{equation}
    \text{Var}(X) = \frac{1}{n} \sum_{i=1}^{n} (x_i - \bar{x})^2
\end{equation}

\subsubsection{Standard Deviation}

The standard deviation is the square root of the variance:

\begin{equation}
    \sigma = \sqrt{\text{Var}(X)}
\end{equation}

\section{Inferential Statistics}

\subsection{Probability Distributions}

\subsubsection{Discrete Probability Distribution}

For a discrete random variable $X$ taking values $x_1, x_2, \ldots, x_n$ with respective probabilities $p_1, p_2, \ldots, p_n$, the probability distribution function is given by:

\begin{equation}
    P(X = x_i) = p_i
\end{equation}

\subsubsection{Continuous Probability Distribution}

For a continuous random variable $X$ with probability density function $f(x)$, the probability of $X$ being in the interval $[a, b]$ is given by:

\begin{equation}
    P(a \leq X \leq b) = \int_{a}^{b} f(x) \,dx
\end{equation}

\subsection{Hypothesis Testing}

\subsubsection{Null Hypothesis ($H_0$)}

The null hypothesis is a statement that there is no significant difference or effect.

\subsubsection{Alternative Hypothesis ($H_1$)}

The alternative hypothesis is a statement indicating that there is a significant difference or effect.

\subsubsection{p-Value}

The p-value is the probability of obtaining test results as extreme as the observed results under the assumption that the null hypothesis is true.

\begin{equation}
    p \leq \alpha \implies \text{Reject } H_0
\end{equation}

\subsection{Confidence Intervals}

A confidence interval is an estimated range of values that is likely to include an unknown population parameter.

\begin{equation}
    \bar{x} \pm z \left(\frac{s}{\sqrt{n}}\right)
\end{equation}

where $\bar{x}$ is the sample mean, $s$ is the sample standard deviation, $n$ is the sample size, and $z$ is the z-score corresponding to the desired confidence level.

\section{Regression Analysis}

\subsection{Linear Regression}

The simple linear regression model relates a dependent variable $Y$ to an independent variable $X$ by the equation:

\begin{equation}
    Y = \beta_0 + \beta_1 X + \epsilon
\end{equation}

where $\beta_0$ is the intercept, $\beta_1$ is the slope, and $\epsilon$ is the error term.

\subsection{Correlation Coefficient}

The correlation coefficient measures the strength and direction of a linear relationship between two variables.

\begin{equation}
    \rho_{XY} = \frac{\text{Cov}(X, Y)}{\sigma_X \sigma_Y}
\end{equation}

\section{Conclusion}

Statistics provides a powerful set of tools for analyzing and interpreting data. The presented methods and concepts offer a foundation for making informed decisions, drawing meaningful conclusions, and gaining insights into the underlying patterns of variability in various fields.

\end{document}