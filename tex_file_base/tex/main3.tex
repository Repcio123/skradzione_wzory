\documentclass{article}
\usepackage{amsmath}

\title{Solving Simple Differential Equations}
\author{Jane Doe}
\date{February 2, 2023}

\begin{document}

\maketitle

\section{Introduction}

Differential equations are fundamental in mathematics and science, describing how quantities change as a function of other variables. In this document, we will explore the basics of solving simple differential equations. We will cover first-order differential equations and second-order differential equations. Let's get started.

\section{First-Order Differential Equations}

A first-order differential equation is an equation involving the derivative of an unknown function, usually denoted as $y(x)$, with respect to the independent variable $x$. A simple first-order differential equation can be written in the form:

\begin{equation}
\frac{dy}{dx} = f(x, y).
\end{equation}

To solve this equation, you can follow these steps:

\begin{enumerate}
  \item Separate variables, if possible, to isolate $y$ on one side and $x$ on the other.
  \item Integrate both sides to find the general solution.
  \item If there are initial conditions, use them to find the particular solution.
\end{enumerate}

For example, let's solve the first-order differential equation:

\begin{equation}
\frac{dy}{dx} = 2x.
\end{equation}

\textbf{Step 1:} Separate variables by moving all terms involving $y$ to one side:

\begin{equation}
\frac{dy}{dx} - 2x = 0.
\end{equation}

\textbf{Step 2:} Integrate both sides:

\begin{equation}
\int \frac{dy}{dx} \, dx - \int 2x \, dx = C,
\end{equation}

where $C$ is the constant of integration.

\begin{equation}
y - x^2 + C = 0.
\end{equation}

\textbf{Step 3:} If given an initial condition, such as $y(0) = 1$, you can solve for $C$:

\begin{equation}
1 - 0^2 + C = 0 \implies C = 1.
\end{equation}

So, the particular solution is:

\begin{equation}
y = x^2 + 1.
\end{equation}

\section{Second-Order Differential Equations}

A second-order differential equation is an equation involving the second derivative of an unknown function, usually denoted as $y(x)$. A simple second-order differential equation can be written in the form:

\begin{equation}
\frac{d^2y}{dx^2} = f(x, y, \frac{dy}{dx}).
\end{equation}

Solving second-order differential equations can be more complex and may require techniques like substitution, separation of variables, or the method of undetermined coefficients.

For example, the second-order differential equation for a simple harmonic oscillator is:

\begin{equation}
\frac{d^2y}{dx^2} + \omega^2 y = 0,
\end{equation}

where $\omega$ is a constant.

The general solution to this equation is:

\begin{equation}
y(x) = A \cos(\omega x) + B \sin(\omega x),
\end{equation}

where $A$ and $B$ are constants determined by initial conditions.

\section{Conclusion}

Solving simple differential equations is a fundamental skill in mathematics and science. First-order and second-order differential equations can be solved using various techniques, as shown in this document. More complex differential equations may require advanced methods, but understanding the basics is a solid foundation for further studies in this field.

This document has provided an introduction to solving simple differential equations, but there is much more to explore in the realm of differential equations and their applications.

\end{document}\textbf{}