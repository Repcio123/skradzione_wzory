\documentclass[12pt]{article}
\usepackage{amsmath, amssymb, amsthm}
\usepackage{tikz}
\usetikzlibrary{positioning, arrows.meta}

\title{Podstawy Teorii Grafów: Definicje i Własności}
\author{Twoje Imię i Nazwisko}
\date{\today}

\begin{document}

\maketitle

\section{Wprowadzenie}
Teoria grafów to dziedzina matematyki zajmująca się badaniem struktury i właściwości grafów. Grafy są abstrakcyjnym modelem reprezentującym zbiór wierzchołków, połączonych krawędziami. W tym artykule omówimy podstawowe definicje związane z grafami oraz przedstawimy kilka kluczowych własności.

\section{Definicje Podstawowe}
\begin{definition}
Graf $G$ składa się z dwóch zbiorów: zbioru wierzchołków $V(G)$ oraz zbioru krawędzi $E(G)$. Krawędź to para wierzchołków, która je łączy.
\end{definition}

\begin{definition}
Graf $G$ jest skierowany, jeśli każda krawędź ma określony kierunek. W przeciwnym razie nazywamy go grafem nieskierowanym.
\end{definition}

\begin{definition}
Stopień wierzchołka to liczba krawędzi incydentnych z danym wierzchołkiem. W grafie nieskierowanym, stopień wierzchołka to liczba krawędzi z nim incydentnych, a w grafie skierowanym, suma liczby krawędzi wchodzących i wychodzących.
\end{definition}

\section{Rodzaje Grafów}
\begin{definition}
Graf pełny, oznaczany jako $K_n$, to graf nieskierowany, w którym każde dwa różne wierzchołki są połączone krawędzią.
\end{definition}

\begin{definition}
Drzewo to spójny graf nieskierowany bez cykli.
\end{definition}

\section{Własności Grafów}
\begin{theorem}
W każdym grafie, suma stopni wszystkich wierzchołków jest równa dwukrotności liczby krawędzi.
\end{theorem}

\begin{proof}
Suma stopni wierzchołków to dwukrotność liczby krawędzi, ponieważ każda krawędź przyczynia się do stopnia dwóch wierzchołków.
\end{proof}

\begin{theorem}
W grafie skierowanym, suma stopni wierzchołków wchodzących jest równa sumie stopni wierzchołków wychodzących.
\end{theorem}

\begin{proof}
W grafie skierowanym każda krawędź przyczynia się do stopnia wierzchołka wchodzącego i wychodzącego, co dowodzi równości sumy stopni wierzchołków wchodzących i wychodzących.
\end{proof}

\section{Podsumowanie}
Teoria grafów jest niezwykle ważna w matematyce i informatyce. W artykule tym omówiliśmy podstawowe definicje grafów, rodzaje grafów oraz przedstawiliśmy dwie kluczowe własności dotyczące stopni wierzchołków. Wiedza ta znajduje zastosowanie w rozmaitych dziedzinach, w tym w sieciach komputerowych, planowaniu tras, czy algorytmach grafowych.

\end{document}
