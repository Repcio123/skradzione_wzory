\documentclass[12pt]{article}
\usepackage{amsmath, amssymb, amsthm}

\title{Podstawy Trygonometrii: Wprowadzenie i Kluczowe Twierdzenia}
\author{Twoje Imię i Nazwisko}
\date{\today}

\begin{document}

\maketitle

\section{Wprowadzenie}
Trygonometria jest gałęzią matematyki, która zajmuje się badaniem relacji między kątami a bokami trójkątów oraz analizą funkcji trygonometrycznych. W tym artykule omówimy podstawowe pojęcia trygonometryczne i przedstawimy kilka kluczowych twierdzeń.

\section{Podstawowe Pojęcia Trygonometryczne}
W trygonometrii wykorzystuje się funkcje trygonometryczne, takie jak sinus, cosinus i tangens. Dla kąta $\theta$ w trójkącie prostokątnym o bokach $a$, $b$ i $c$ (gdzie $c$ to przeciwprostokątna), definiujemy:

\begin{align*}
\sin(\theta) &= \frac{a}{c} \\
\cos(\theta) &= \frac{b}{c} \\
\tan(\theta) &= \frac{a}{b}
\end{align*}

\section{Twierdzenie o Sumie Kątów}
Jednym z kluczowych twierdzeń trygonometrycznych jest twierdzenie o sumie kątów, które mówi, że suma sinusów (cosinusów) dwóch kątów jest równa sinusowi (cosinusowi) sumy tych kątów.

\begin{theorem}
Dla dowolnych kątów $\alpha$ i $\beta$:
\[
\sin(\alpha + \beta) = \sin(\alpha)\cos(\beta) + \cos(\alpha)\sin(\beta)
\]
\[
\cos(\alpha + \beta) = \cos(\alpha)\cos(\beta) - \sin(\alpha)\sin(\beta)
\]
\end{theorem}

\section{Twierdzenie Pitagorasa}
Twierdzenie Pitagorasa stanowi fundamentalną zasadę w trygonometrii, łączącą długości boków trójkąta prostokątnego.

\begin{theorem}
W trójkącie prostokątnym o bokach $a$, $b$ i $c$ (gdzie $c$ to przeciwprostokątna), zachodzi równość:
\[
a^2 + b^2 = c^2
\]
\end{theorem}

\section{Podsumowanie}
Trygonometria jest niezwykle istotną dziedziną matematyki, znalezienie zastosowania m.in. w fizyce, inżynierii i wielu innych dziedzinach nauki. W tym artykule omówiliśmy podstawowe pojęcia trygonometryczne oraz przedstawiliśmy dwie kluczowe zasady: twierdzenie o sumie kątów oraz twierdzenie Pitagorasa.

\end{document}
