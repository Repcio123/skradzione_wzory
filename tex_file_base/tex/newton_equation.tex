\documentclass[12pt]{article}
\usepackage{amsmath, amssymb, amsthm}

\title{Równanie Newtona: Metody Iteracyjne i Zastosowania}
\author{Twoje Imię i Nazwisko}
\date{\today}

\begin{document}

\maketitle

\section{Wprowadzenie}
Równanie Newtona to jedno z fundamentalnych równań matematycznych, używane do znajdowania miejsc zerowych funkcji. W tym artykule omówimy metody iteracyjne rozwiązywania równania Newtona, jego własności oraz zastosowania w analizie numerycznej.

\section{Równanie Newtona}
\begin{definition}
Równanie Newtona dla funkcji rzeczywistej $f(x)$ ma postać:
\[
f(x) = 0
\]
Metoda Newtona polega na iteracyjnym przybliżaniu miejsc zerowych tej funkcji.
\end{definition}

\section{Metoda Iteracyjna Newtona}
\begin{theorem}
Metoda Newtona zakłada iteracyjne stosowanie wzoru:
\[
x_{n+1} = x_n - \frac{f(x_n)}{f'(x_n)}
\]
gdzie $f'(x_n)$ to pochodna funkcji $f(x)$ w punkcie $x_n$.
\end{theorem}

\section{Własności i Warunki Zbieżności}
\begin{itemize}
  \item \textbf{Warunek Zbieżności:} Metoda Newtona zwykle zbiega szybko do rozwiązania, jeśli startujemy z dostatecznie bliskiego punktu.
  
  \item \textbf{Złożoność Obliczeniowa:} Obliczenia wymagają obliczenia pochodnych, co może być czasochłonne.
  
  \item \textbf{Stabilność Numeryczna:} Metoda może być niestabilna w przypadku, gdy jest stosowana w okolicach ekstremalnych.
\end{itemize}

\section{Zastosowania}
\begin{itemize}
  \item \textbf{Analiza Numeryczna:} Metoda Newtona jest powszechnie używana do rozwiązywania równań nieliniowych w analizie numerycznej.
  
  \item \textbf{Optymalizacja:} Równanie Newtona jest wykorzystywane w algorytmach optymalizacji do znajdowania punktów minimalnych i maksymalnych funkcji.
  
  \item \textbf{Fizyka i Inżynieria:} W różnych dziedzinach nauki i technologii, równanie Newtona jest stosowane w rozwiązywaniu problemów matematycznych opisujących zjawiska fizyczne.
\end{itemize}

\section{Podsumowanie}
Równanie Newtona i metoda iteracyjna, której używamy do jego rozwiązywania, są kluczowe w analizie numerycznej i optymalizacji. W artykule tym omówiliśmy podstawowe definicje, metody iteracyjne, własności oraz zastosowania równania Newtona w różnych dziedzinach matematyki i nauki.

\end{document}
