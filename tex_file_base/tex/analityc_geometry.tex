\documentclass{article}
\usepackage{amsmath}
\usepackage{amsfonts}
\usepackage{amssymb}
\usepackage{geometry}

\geometry{a4paper, margin=1in}

\title{Analytic Geometry: Exploring Mathematical Spaces}
\author{Your Name}
\date{\today}

\begin{document}

\maketitle

\section{Introduction}

Analytic geometry is a branch of mathematics that combines algebra and geometry, allowing us to study geometric shapes using algebraic methods. This approach provides a powerful toolkit for understanding and solving problems in various fields.

\section{Coordinate Systems}

\subsection{Cartesian Coordinates}

In a Cartesian coordinate system, points in the plane are represented by ordered pairs $(x, y)$. The distance between two points $P(x_1, y_1)$ and $Q(x_2, y_2)$ is given by the distance formula:

\begin{equation}
    d(P, Q) = \sqrt{(x_2 - x_1)^2 + (y_2 - y_1)^2}
\end{equation}

\subsection{Polar Coordinates}

Polar coordinates represent points in the plane using the distance $r$ from the origin and the angle $\theta$ with the positive $x$-axis. Conversion between Cartesian and polar coordinates is given by:

\begin{align}
    x &= r \cos \theta \\
    y &= r \sin \theta
\end{align}

\section{Equations of Lines}

\subsection{Slope-Intercept Form}

The equation of a line in slope-intercept form is given by:

\begin{equation}
    y = mx + b
\end{equation}

where $m$ is the slope and $b$ is the $y$-intercept.

\subsection{Point-Slope Form}

The point-slope form of a line is given by:

\begin{equation}
    y - y_1 = m(x - x_1)
\end{equation}

where $(x_1, y_1)$ is a point on the line.

\subsection{Two-Point Form}

The equation of a line passing through two points $(x_1, y_1)$ and $(x_2, y_2)$ is given by:

\begin{equation}
    \frac{y - y_1}{y_2 - y_1} = \frac{x - x_1}{x_2 - x_1}
\end{equation}

\section{Conic Sections}

\subsection{Circle}

The equation of a circle with center $(h, k)$ and radius $r$ is given by:

\begin{equation}
    (x - h)^2 + (y - k)^2 = r^2
\end{equation}

\subsection{Ellipse}

The equation of an ellipse centered at the origin with semi-major axis $a$ and semi-minor axis $b$ is given by:

\begin{equation}
    \frac{x^2}{a^2} + \frac{y^2}{b^2} = 1
\end{equation}

\subsection{Hyperbola}

The equation of a hyperbola centered at the origin with vertices on the $x$-axis is given by:

\begin{equation}
    \frac{x^2}{a^2} - \frac{y^2}{b^2} = 1
\end{equation}

\subsection{Parabola}

The equation of a parabola with focus at $(h, k)$ and axis parallel to the $x$-axis is given by:

\begin{equation}
    (x - h)^2 = 4p(y - k)
\end{equation}

\section{Transformations}

\subsection{Translation}

A translation of a point $(x, y)$ by $(a, b)$ results in a new point $(x + a, y + b)$.

\subsection{Rotation}

A rotation of a point $(x, y)$ by an angle $\theta$ about the origin is given by:

\begin{align}
    x' &= x \cos \theta - y \sin \theta \\
    y' &= x \sin \theta + y \cos \theta
\end{align}

\subsection{Scaling}

A scaling of a point $(x, y)$ by factors $c_1$ and $c_2$ along the $x$ and $y$ axes, respectively, is given by:

\begin{align}
    x' &= c_1 x \\
    y' &= c_2 y
\end{align}

\section{Three-Dimensional Space}

Analytic geometry extends to three-dimensional space, where points are represented by ordered triples $(x, y, z)$.

\subsection{Equation of a Plane}

The equation of a plane in three-dimensional space is given by:

\begin{equation}
    ax + by + cz = d
\end{equation}

where $(a, b, c)$ is the normal vector to the plane.

\section{Conclusion}

Analytic geometry provides a powerful set of tools for studying geometric shapes using algebraic techniques. The presented formulas and concepts offer a foundation for exploring mathematical spaces in various dimensions and solving real-world problems.

\end{document}