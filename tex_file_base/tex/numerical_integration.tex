\documentclass{article}
\usepackage{amsmath}
\usepackage{amsfonts}
\usepackage{amssymb}
\usepackage{geometry}

\geometry{a4paper, margin=1in}

\title{Numerical Integration: Methods and Applications}
\author{Your Name}
\date{\today}

\begin{document}

\maketitle

\section{Introduction}

Numerical integration is a fundamental technique in computational mathematics, employed to approximate definite integrals of functions that may be difficult or impossible to solve analytically. This field plays a crucial role in various scientific and engineering applications.

\section{Basic Concepts}

\subsection{Trapezoidal Rule}

The trapezoidal rule is a simple method to approximate the definite integral of a function $f(x)$ over an interval $[a, b]$:

\begin{equation}
    \int_{a}^{b} f(x) \,dx \approx \frac{b - a}{2} \left[f(a) + f(b)\right]
\end{equation}

\subsection{Simpson's Rule}

Simpson's rule provides a more accurate approximation using quadratic polynomials:

\begin{equation}
    \int_{a}^{b} f(x) \,dx \approx \frac{b - a}{6} \left[f(a) + 4f\left(\frac{a+b}{2}\right) + f(b)\right]
\end{equation}

\subsection{Midpoint Rule}

The midpoint rule uses the midpoint of each subinterval for approximation:

\begin{equation}
    \int_{a}^{b} f(x) \,dx \approx h \sum_{i=1}^{n} f\left(a + \frac{h}{2} + ih\right)
\end{equation}

where $h = \frac{b - a}{n}$ and $n$ is the number of subintervals.

\section{Composite Numerical Integration}

To improve accuracy, composite methods combine multiple applications of basic rules over smaller subintervals. The composite trapezoidal rule, composite Simpson's rule, and composite midpoint rule are commonly used.

\subsection{Composite Trapezoidal Rule}

\begin{equation}
    \int_{a}^{b} f(x) \,dx \approx \frac{h}{2} \left[f(a) + 2\sum_{i=1}^{n-1} f(a + ih) + f(b)\right]
\end{equation}

\subsection{Composite Simpson's Rule}

\begin{equation}
    \int_{a}^{b} f(x) \,dx \approx \frac{h}{3} \left[f(a) + 4\sum_{i=1}^{n-1} f\left(a + \frac{2ih}{2}\right) + 2\sum_{i=1}^{n} f(a + ih) + f(b)\right]
\end{equation}

\subsection{Composite Midpoint Rule}

\begin{equation}
    \int_{a}^{b} f(x) \,dx \approx h \sum_{j=1}^{m} \sum_{i=1}^{n} f\left(a + \frac{h}{2} + (i-1)h, c_j\right)
\end{equation}

where $h = \frac{b - a}{n}$, $n$ is the number of subintervals, and $c_j$ is the midpoint of each subinterval.

\section{Error Analysis}

The error in numerical integration methods can be estimated using techniques such as Richardson extrapolation and asymptotic error expansions.

\subsection{Richardson Extrapolation}

Richardson extrapolation is a technique to improve the accuracy of numerical methods by combining multiple approximations with different step sizes.

\begin{equation}
    I_h = I + c_1h^p + c_2h^{2p} + \ldots
\end{equation}

where $I_h$ is the numerical approximation with step size $h$, and $p$ is the order of convergence.

\section{Adaptive Quadrature}

Adaptive quadrature methods dynamically adjust the step size based on the function's behavior to achieve accurate results efficiently.

\subsection{Adaptive Simpson's Rule}

Adaptive Simpson's rule adapts the step size based on the difference between the Simpson's rule approximation and the composite approximation with smaller subintervals.

\section{Conclusion}

Numerical integration methods play a crucial role in approximating definite integrals, allowing for the efficient evaluation of mathematical expressions in various scientific and engineering applications. The presented formulas and techniques provide a foundation for understanding and implementing numerical integration algorithms.

\end{document}