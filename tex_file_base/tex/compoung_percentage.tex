\documentclass[12pt]{article}
\usepackage{amsmath, amssymb, amsthm}

\title{Procent Składany: Definicje, Wzory i Zastosowania}
\author{Twoje Imię i Nazwisko}
\date{\today}

\begin{document}

\maketitle

\section{Wprowadzenie}
Procent składany to koncepcja matematyczna związana z rosnącym saldem w wyniku regularnych odsetek na kapitale początkowym. W tym artykule omówimy definicję procentu składanego, przedstawimy wzory obliczeniowe oraz zastosowania w finansach i ekonomii.

\section{Definicja Procentu Składanego}
\begin{definition}
Procent składany to proces, w którym odsetki zainwestowane w danym okresie również generują odsetki w kolejnych okresach. Kapitał początkowy rośnie eksponencjalnie.
\end{definition}

\section{Wzór Ogólny na Procent Składany}
\begin{theorem}
Ogólny wzór na procent składany można przedstawić jako:
\[
A = P \left(1 + \frac{r}{n}\right)^{nt}
\]
gdzie:
\begin{itemize}
  \item $A$ to końcowy kapitał,
  \item $P$ to początkowy kapitał,
  \item $r$ to stopa procentowa,
  \item $n$ to liczba składanych odsetek w ciągu roku,
  \item $t$ to czas trwania inwestycji w latach.
\end{itemize}
\end{theorem}

\section{Wzory Particularne}
\begin{itemize}
  \item \textbf{Procent Składany Roczny:} $A = P(1 + r)^t$
  
  \item \textbf{Procent Składany Miesięczny:} $A = P\left(1 + \frac{r}{12}\right)^{12t}$
  
  \item \textbf{Procent Składany Dzienny:} $A = P\left(1 + \frac{r}{365}\right)^{365t}$
\end{itemize}

\section{Zastosowania w Finansach}
\begin{itemize}
  \item \textbf{Inwestycje:} Procent składany jest kluczowy w obliczeniach zysków z inwestycji, gdzie odsetki generują kolejne odsetki.
  
  \item \textbf{Kredyty:} W przypadku kredytów, procent składany może określać łączną kwotę do spłaty po pewnym czasie.
  
  \item \textbf{Oszczędności:} Procent składany jest istotny przy określaniu wartości przyszłych oszczędności.
\end{itemize}

\section{Podsumowanie}
Procent składany jest ważnym zagadnieniem matematycznym w finansach i ekonomii. W artykule tym omówiliśmy definicję procentu składanego, przedstawiliśmy ogólny wzór oraz szczególne przypadki, a także zastosowania tej koncepcji w różnych dziedzinach.

\end{document}
