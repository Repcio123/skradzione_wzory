\documentclass[12pt]{article}
\usepackage{amsmath, amssymb, amsthm}

\title{Całkowanie Numeryczne: Metody, Błędy i Zastosowania}
\author{Twoje Imię i Nazwisko}
\date{\today}

\begin{document}

\maketitle

\section{Wprowadzenie}
Całkowanie numeryczne to gałąź analizy numerycznej zajmująca się przybliżonym obliczaniem wartości całek definowanych na funkcjach. Jest niezwykle ważne w wielu dziedzinach matematyki, fizyki, inżynierii i nauk przyrodniczych. W tym artykule omówimy podstawowe metody całkowania numerycznego, analizę błędów oraz zastosowania praktyczne.

\section{Definicje Podstawowe}
\begin{definition}
Rozważmy funkcję $f(x)$ określoną na przedziale $[a, b]$. Całkowanie numeryczne polega na przybliżeniu wartości całki $\int_{a}^{b} f(x) \,dx$ za pomocą skończonej liczby operacji arytmetycznych.
\end{definition}

\section{Metody Całkowania Numerycznego}
\begin{enumerate}
  \item \textbf{Metoda Prostokątów:} Polega na przybliżeniu obszaru pod krzywą za pomocą prostokątów i sumowaniu ich pól.
  
  \item \textbf{Metoda Trapezów:} Wykorzystuje trapezy jako przybliżenie obszaru pod krzywą, co daje dokładniejsze wyniki niż metoda prostokątów.
  
  \item \textbf{Metoda Simpsona:} Stosuje parabolę jako przybliżenie obszaru, co jest bardziej dokładne niż metoda trapezów, szczególnie dla funkcji gładkich.
  
  \item \textbf{Kwadratura Gaussa:} Wykorzystuje specjalne wagi i punkty, aby uzyskać jeszcze dokładniejsze przybliżenia całek.
\end{enumerate}

\section{Analiza Błędów}
\begin{theorem}
Błąd całkowania numerycznego zależy od rodzaju metody oraz liczby podprzedziałów użytych do przybliżenia całki.
\end{theorem}

\begin{theorem}
Błędy zaokrągleń mogą wpływać na dokładność wyników całkowania numerycznego, zwłaszcza dla skomplikowanych funkcji.
\end{theorem}

\section{Zastosowania Całkowania Numerycznego}
\begin{itemize}
  \item \textbf{Analiza numeryczna funkcji:} Całkowanie numeryczne jest szeroko stosowane w analizie funkcji, zwłaszcza gdy brak jest analitycznych rozwiązań całek.
  
  \item \textbf{Rozwiązywanie równań różniczkowych:} W wielu numerycznych metodach rozwiązywania równań różniczkowych, całkowanie numeryczne odgrywa kluczową rolę.
  
  \item \textbf{Symulacje numeryczne:} W dziedzinie nauk przyrodniczych, fizyki czy inżynierii, całkowanie numeryczne jest niezbędne do przeprowadzania dokładnych symulacji numerycznych.
\end{itemize}

\section{Podsumowanie}
Całkowanie numeryczne to istotna dziedzina analizy numerycznej, znajdująca zastosowanie w różnych dziedzinach matematyki i nauk przyrodniczych. W artykule tym omówiliśmy podstawowe metody całkowania numerycznego, analizę błędów oraz praktyczne zastosowania tej techniki.

\end{document}
