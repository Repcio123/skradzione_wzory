\documentclass[12pt]{article}
\usepackage{amsmath, amssymb, amsthm}

\title{Równanie Kwadratowe: Własności i Zastosowania}
\author{Twoje Imię i Nazwisko}
\date{\today}

\begin{document}

\maketitle

\section{Wprowadzenie}
Równanie kwadratowe to jedno z podstawowych równań kwadratowych w matematyce. Jego ogólna postać to $ax^2 + bx + c = 0$, gdzie $a$, $b$ i $c$ są stałymi, a $x$ jest zmienną. W tym artykule omówimy własności równania kwadratowego oraz zastosowania tego typu równań.

\section{Równanie Kwadratowe: Postać Kanoniczna}
\begin{definition}
Równanie kwadratowe w postaci kanonicznej ma postać:
\[
ax^2 + bx + c = a\left(x + \frac{b}{2a}\right)^2 - \frac{b^2 - 4ac}{4a}
\]
\end{definition}

\section{Własności Delta}
\begin{definition}
Delta ($\Delta$) równania kwadratowego $ax^2 + bx + c = 0$ to wyrażenie $b^2 - 4ac$. Własności delty decydują o charakterze pierwiastków równania:
\begin{itemize}
  \item $\Delta > 0$: Dwa pierwiastki rzeczywiste.
  \item $\Delta = 0$: Jeden pierwiastek rzeczywisty (mający krotność 2).
  \item $\Delta < 0$: Dwa pierwiastki zespolone (koniektyczne).
\end{itemize}
\end{definition}

\section{Rozwiązania Równania Kwadratowego}
\begin{theorem}
Dla równania kwadratowego $ax^2 + bx + c = 0$, rozwiązania $x$ są dawane wzorem kwadratowym:
\[
x_{1,2} = \frac{-b \pm \sqrt{\Delta}}{2a}
\]
\end{theorem}

\section{Zastosowania w Analizie Matematycznej}
\begin{itemize}
  \item \textbf{Fizyka:} Równania kwadratowe pojawiają się w zadaniach związanych z ruchem jednostajnym przyspieszonym.
  
  \item \textbf{Ekonomia:} W analizie ekonomicznej równania kwadratowe mają zastosowanie w modelach prognozowania.
  
  \item \textbf{Inżynieria:} W inżynierii równania kwadratowe mogą opisywać relacje pomiędzy różnymi zmiennymi.
\end{itemize}

\section{Podsumowanie}
Równanie kwadratowe to kluczowe zagadnienie w matematyce, z wieloma zastosowaniami praktycznymi. W artykule tym omówiliśmy jego postać kanoniczną, własności delty, wzory kwadratowe oraz zastosowania w różnych dziedzinach nauki i technologii.

\end{document}
