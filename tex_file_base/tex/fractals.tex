\documentclass[12pt]{article}
\usepackage{graphicx}
\usepackage{amsmath, amssymb, amsthm}
\usepackage{tikz}

\title{Fraktale: Definicje, Właściwości i Zastosowania}
\author{Twoje Imię i Nazwisko}
\date{\today}

\begin{document}

\maketitle

\section{Wprowadzenie}
Fraktale to struktury matematyczne charakteryzujące się samo-podobieństwem na różnych skalach. Ich niezwykłe i skomplikowane wzory powtarzają się w nieskończoność, tworząc interesujące i piękne formy. W tym artykule omówimy podstawowe definicje związane z fraktalami oraz przedstawimy kilka kluczowych właściwości.

\section{Definicje Podstawowe}
\begin{definition}
Fraktal to matematyczny obiekt lub zbiór, który jest samo-podobny na różnych skalach. Oznacza to, że fragmenty fraktala wyglądają podobnie do całości.
\end{definition}

\begin{definition}
Iteracyjny proces tworzenia fraktali polega na wielokrotnym powtarzaniu pewnego wzoru lub reguły, zwykle z wykorzystaniem komputera.
\end{definition}

\section{Przykłady Fraktali}
\begin{itemize}
  \item \textbf{Zbiór Julia:} Matematyczny zbiór punktów w płaszczyźnie zdefiniowany przez iteracyjne stosowanie funkcji.
  \item \textbf{Zbiór Mandelbrota:} Fraktal powstający w procesie iteracyjnego wykonywania prostej operacji matematycznej na liczbach zespolonych.
  \item \textbf{Kurwa Peano:} Jedna z krzywych Peano, które są fraktalnymi krzywymi wypełniającymi płaszczyznę.
\end{itemize}

\section{Właściwości Fraktali}
\begin{theorem}
Fraktale posiadają wymiar fraktalny, który może być ułamkiem. Jest to wskaźnik skomplikowania struktury fraktala.
\end{theorem}

\begin{theorem}
Fraktale są nieprzerywane i nieskończenie złożone nawet przy skończonej długości lub powierzchni.
\end{theorem}

\section{Zastosowania Fraktali}
\begin{itemize}
  \item \textbf{Grafika komputerowa:} Fraktale są używane do generowania realistycznych i ciekawych tekstur.
  \item \textbf{Komputerowe modele terenów:} W geografii komputerowej, fraktale są stosowane do generowania realistycznych krajobrazów.
  \item \textbf{Teoria chaosu:} Fraktale są powiązane z teorią chaosu, a ich badanie pomaga zrozumieć złożoność nieliniowych systemów.
\end{itemize}

\section{Podsumowanie}
Fraktale stanowią fascynujące i piękne obiekty matematyczne o skomplikowanej strukturze. W artykule tym omówiliśmy podstawowe definicje związane z fraktalami, przedstawiliśmy kilka przykładów oraz omówiliśmy ich właściwości i zastosowania. Zrozumienie fraktali ma zastosowanie w wielu dziedzinach, od matematyki po sztuczną inteligencję. 

\end{document}
