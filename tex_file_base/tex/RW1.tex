\documentclass{article}
\usepackage{amsmath} % Pakiet do wzorów matematycznych

\title{Exploring Mathematical Formulas}
\author{Your Name}
\date{\today}

\begin{document}

\maketitle

\section{Introduction}
In this article, we will explore various mathematical formulas and their applications. We'll cover a range of mathematical concepts and equations.

\section{Quadratic Equations}
One of the most well-known equations in algebra is the quadratic equation. It's expressed as:

\begin{equation}
    ax^2 + bx + c = 0
\end{equation}

The solutions for $x$ can be found using the quadratic formula:

\begin{equation}
    x = \frac{-b \pm \sqrt{b^2 - 4ac}}{2a}
\end{equation}

\section{Euler's Identity}
Euler's identity is a famous mathematical formula that combines several important mathematical constants:

\begin{equation}
    e^{i\pi} + 1 = 0
\end{equation}

This equation relates the mathematical constants $e$, $i$, $\pi$, 1, and 0 in an elegant and surprising way.

\section{The Pythagorean Theorem}
The Pythagorean Theorem is a fundamental concept in geometry. It states that in a right triangle, the sum of the squares of the two shorter sides is equal to the square of the length of the hypotenuse. The theorem is represented as:

\begin{equation}
    a^2 + b^2 = c^2
\end{equation}

\section{The Fundamental Theorem of Calculus}
The Fundamental Theorem of Calculus is a significant result in calculus. It relates differentiation and integration and is expressed as:

\begin{equation}
    \int_a^b f(x) \, dx = F(b) - F(a)
\end{equation}

where $F(x)$ is the antiderivative of $f(x)$.

\section{Conclusion}
Mathematics is a rich and diverse field with a wide range of formulas and theorems. The equations we've explored in this article are just a small sample of the many mathematical concepts that play a crucial role in various scientific and practical applications.

\end{document}
