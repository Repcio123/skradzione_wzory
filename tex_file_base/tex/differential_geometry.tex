\documentclass{article}
\usepackage{amsmath}
\usepackage{amsfonts}
\usepackage{amssymb}
\usepackage{geometry}

\geometry{a4paper, margin=1in}

\title{Differential Geometry: A Mathematical Overview}
\author{Your Name}
\date{\today}

\begin{document}

\maketitle

\section{Introduction}

Differential geometry is a branch of mathematics that deals with the study of geometric objects using differential and integral calculus. It plays a crucial role in various areas such as physics, computer science, and engineering.

\section{Basic Concepts}

\subsection{Curves in Space}

Consider a parametrically defined curve $\mathbf{r}(t) = (x(t), y(t), z(t))$ in three-dimensional space. The velocity vector $\mathbf{v}(t)$ and acceleration vector $\mathbf{a}(t)$ are given by:

\begin{align}
    \mathbf{v}(t) &= \frac{d\mathbf{r}}{dt} = (x'(t), y'(t), z'(t)) \\
    \mathbf{a}(t) &= \frac{d\mathbf{v}}{dt} = (x''(t), y''(t), z''(t))
\end{align}

\subsection{Tangent Vectors}

The tangent vector to a curve is obtained by normalizing the velocity vector:

\begin{equation}
    \mathbf{T}(t) = \frac{\mathbf{v}(t)}{\|\mathbf{v}(t)\|}
\end{equation}

\subsection{Curvature}

The curvature of a curve is defined as:

\begin{equation}
    \kappa(t) = \frac{\|\mathbf{v}(t) \times \mathbf{a}(t)\|}{\|\mathbf{v}(t)\|^3}
\end{equation}

\section{Surfaces in Space}

\subsection{Parametric Surfaces}

A parametrically defined surface $\mathbf{S}(u, v) = (x(u, v), y(u, v), z(u, v))$ can be represented in terms of two parameters.

\subsection{Normal Vector}

The normal vector to a surface is given by the cross product of the tangent vectors:

\begin{equation}
    \mathbf{N}(u, v) = \frac{\partial \mathbf{S}}{\partial u} \times \frac{\partial \mathbf{S}}{\partial v}
\end{equation}

\subsection{Gaussian Curvature}

The Gaussian curvature $K$ of a surface is defined as:

\begin{equation}
    K = \frac{\det(\mathbf{S}_u \times \mathbf{S}_v)}{\|\mathbf{S}_u \times \mathbf{S}_v\|^2}
\end{equation}

\section{The Fundamental Forms}

\subsection{First Fundamental Form}

The first fundamental form of a surface is given by:

\begin{equation}
    I = E du^2 + 2F du dv + G dv^2
\end{equation}

where $E$, $F$, and $G$ are coefficients derived from the metric tensor.

\subsection{Second Fundamental Form}

The second fundamental form is defined as:

\begin{equation}
    II = L du^2 + 2M du dv + N dv^2
\end{equation}

where $L$, $M$, and $N$ are coefficients related to the shape of the surface.

\section{Geodesics}

Geodesics are curves that locally minimize distance. The geodesic equation is given by:

\begin{equation}
    \frac{d^2 u^i}{ds^2} + \Gamma_{jk}^i \frac{du^j}{ds} \frac{du^k}{ds} = 0
\end{equation}

where $\Gamma_{jk}^i$ are the Christoffel symbols.

\section{Conclusion}

Differential geometry provides a powerful framework for understanding the geometry of curves and surfaces. The presented formulas offer a glimpse into the rich mathematical structure underlying this fascinating field.

\end{document}