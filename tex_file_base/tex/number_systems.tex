\documentclass{article}
\usepackage{amsmath}
\usepackage{amsfonts}
\usepackage{amssymb}
\usepackage{geometry}

\geometry{a4paper, margin=1in}

\title{Number Systems: A Comprehensive Study}
\author{Your Name}
\date{\today}

\begin{document}

\maketitle

\section{Introduction}

Number systems are fundamental mathematical structures used to represent and manipulate numerical quantities. This paper explores various number systems, their properties, and their applications in different mathematical contexts.

\section{Decimal System}

\subsection{Basic Concepts}

The decimal system, also known as the base-10 system, uses digits 0 through 9 to represent numbers. The value of a number is determined by the position of its digits.

\subsection{Place Value}

In the decimal system, the place value of a digit is based on powers of 10. For example, in the number $3482.16$, the digit $8$ is in the hundreds place, representing $8 \times 10^2$.

\section{Binary System}

\subsection{Representation}

The binary system, or base-2 system, uses only the digits 0 and 1. Each digit's place value is a power of 2. For example, the binary number $101101$ represents:

\begin{equation}
    1 \times 2^5 + 0 \times 2^4 + 1 \times 2^3 + 1 \times 2^2 + 0 \times 2^1 + 1 \times 2^0 = 45_{10}
\end{equation}

\subsection{Conversion}

To convert a decimal number to binary, the number is successively divided by 2, and the remainders, read in reverse order, give the binary representation.

\section{Octal and Hexadecimal Systems}

\subsection{Octal System}

The octal system, or base-8 system, uses digits 0 through 7. It is commonly used in computing. For example, the octal number $36_8$ represents:

\begin{equation}
    3 \times 8^1 + 6 \times 8^0 = 30_{10}
\end{equation}

\subsection{Hexadecimal System}

The hexadecimal system, or base-16 system, uses digits 0 through 9 and letters A through F to represent values from 10 to 15. It is widely used in computing to represent binary-coded values conveniently. For example, the hexadecimal number $1A3_{16}$ represents:

\begin{equation}
    1 \times 16^2 + 10 \times 16^1 + 3 \times 16^0 = 419_{10}
\end{equation}

\section{Conversion Between Bases}

\subsection{Decimal to Binary}

To convert a decimal number to binary, repeatedly divide by 2 and record the remainders.

\subsection{Binary to Decimal}

To convert a binary number to decimal, sum the products of each bit by its place value.

\subsection{Binary to Hexadecimal}

Group binary digits into sets of four and replace each set with its corresponding hexadecimal digit.

\subsection{Hexadecimal to Binary}

Replace each hexadecimal digit with its equivalent 4-bit binary representation.

\section{Positional Notation}

Positional notation is a general concept that extends to different bases. In base-$b$, a number is represented as a series of digits $d_n d_{n-1} \ldots d_1 d_0$, where each $d_i$ is multiplied by $b^i$.

\section{Complex Number Systems}

\subsection{Real Numbers}

The real number system includes all rational and irrational numbers, represented on the number line.

\subsection{Imaginary Numbers}

The imaginary unit $i$ is defined such that $i^2 = -1$. Imaginary numbers are expressed as $a \cdot i$, where $a$ is a real number.

\subsection{Complex Numbers}

Complex numbers are expressed in the form $a + bi$, where $a$ and $b$ are real numbers.

\section{Conclusion}

Number systems are essential in various mathematical and computational applications. The presented concepts and methods provide a foundation for understanding different number systems and their role in representing and manipulating numerical quantities.

\end{document}