\documentclass{article}
\usepackage{amsmath}
\usepackage{amsfonts}
\usepackage{amssymb}
\usepackage{geometry}

\geometry{a4paper, margin=1in}

\title{Trygonometria: Analiza podstawowych funkcji trygonometrycznych}
\author{Twoje Imię i Nazwisko}
\date{\today}

\begin{document}

\maketitle

\section{Wprowadzenie}

Trygonometria jest działem matematyki zajmującym się badaniem relacji między kątami a długościami boków w trójkątach. Podstawowe funkcje trygonometryczne, takie jak sinus, cosinus i tangens, są niezwykle istotne w wielu dziedzinach nauki i inżynierii.

\section{Podstawowe wzory trygonometryczne}

\subsection{Równania podstawowe}

\begin{align}
    \sin^2 \theta + \cos^2 \theta &= 1 \\
    \tan \theta &= \frac{\sin \theta}{\cos \theta} \\
    \cot \theta &= \frac{1}{\tan \theta} = \frac{\cos \theta}{\sin \theta}
\end{align}

\subsection{Wzory sumy i różnicy}

\begin{align}
    \sin(A + B) &= \sin A \cos B + \cos A \sin B \\
    \cos(A + B) &= \cos A \cos B - \sin A \sin B \\
    \tan(A + B) &= \frac{\tan A + \tan B}{1 - \tan A \tan B}
\end{align}

\section{Funkcje trygonometryczne w układzie współrzędnych}

\subsection{Definicje funkcji trygonometrycznych}

W układzie współrzędnych, dla punktu $(x, y)$ na okręgu jednostkowym o promieniu $r=1$, funkcje trygonometryczne są zdefiniowane jako:

\begin{align}
    \sin \theta &= y \\
    \cos \theta &= x \\
    \tan \theta &= \frac{y}{x}
\end{align}

\subsection{Własności funkcji trygonometrycznych}

\begin{align}
    \sin(\theta + 2\pi n) &= \sin \theta \\
    \cos(\theta + 2\pi n) &= \cos \theta \\
    \tan(\theta + \pi n) &= \tan \theta
\end{align}

\section{Wzory redukcyjne}

\begin{align}
    \sin(-\theta) &= -\sin \theta \\
    \cos(-\theta) &= \cos \theta \\
    \tan(-\theta) &= -\tan \theta \\
    \sin(180^\circ - \theta) &= \sin \theta \\
    \cos(180^\circ - \theta) &= -\cos \theta \\
    \tan(180^\circ - \theta) &= -\tan \theta
\end{align}

\section{Wzory potrójne}

\begin{align}
    \sin(3\theta) &= 3\sin \theta - 4\sin^3 \theta \\
    \cos(3\theta) &= 4\cos^3 \theta - 3\cos \theta \\
    \tan(3\theta) &= \frac{3\tan \theta - \tan^3 \theta}{1 - 3\tan^2 \theta}
\end{align}

\section{Podsumowanie}

Trygonometria jest niezwykle ważnym obszarem matematyki, zastosowanym w wielu dziedzinach nauki i inżynierii. Przedstawione wzory stanowią jedynie wstęp do głębszego zrozumienia tej fascynującej dziedziny.

\end{document}