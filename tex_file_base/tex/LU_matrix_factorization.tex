\documentclass[12pt]{article}
\usepackage{amsmath, amssymb, amsthm}

\title{Faktoryzacja LU Macierzy: Teoria, Algorytmy i Zastosowania}
\author{Twoje Imię i Nazwisko}
\date{\today}

\begin{document}

\maketitle

\section{Wprowadzenie}
Faktoryzacja LU (Lower-Upper) macierzy to jedna z fundamentalnych operacji w algebrze liniowej. Polega na przedstawieniu danej macierzy jako iloczyn macierzy dolnotrójkątnej (Lower) i górnotrójkątnej (Upper). W tym artykule omówimy teorię faktoryzacji LU, przedstawimy algorytmy jej obliczania oraz zastosowania praktyczne.

\section{Definicje Podstawowe}
\begin{definition}
Niech $A$ będzie macierzą wymiaru $n \times n$. Faktoryzacja LU polega na znalezieniu dwóch macierzy $L$ (macierz dolnotrójkątna) i $U$ (macierz górnotrójkątna) takich, że $A = LU$.
\end{definition}

\section{Algorytmy Faktoryzacji LU}
\begin{enumerate}
  \item \textbf{Metoda Eliminacji Gaussa:} Klasyczna metoda polegająca na eliminacji zmiennych w układzie równań, co prowadzi do uzyskania macierzy w postaci faktoryzacji LU.
  
  \item \textbf{Metoda Doolittle'a:} Algorytm, który bezpośrednio oblicza macierze $L$ i $U$ na podstawie eliminacji Gaussa.
  
  \item \textbf{Metoda Crouta:} Inna wersja bezpośredniego algorytmu faktoryzacji LU, w której różnice między $L$ i $U$ są minimalizowane.
\end{enumerate}

\section{Własności Faktoryzacji LU}
\begin{theorem}
Faktoryzacja LU macierzy istnieje dla macierzy kwadratowej $A$ o pełnym rzędzie (wszystkie minory główne są nieosobliwe).
\end{theorem}

\begin{theorem}
Faktoryzacja LU macierzy jest niejednoznaczna, jednak dla pewnych warunków macierze $L$ i $U$ mogą być jednoznacznie określone.
\end{theorem}

\section{Zastosowania Faktoryzacji LU}
\begin{itemize}
  \item \textbf{Rozwiązywanie układów równań:} Faktoryzacja LU jest często stosowana do efektywnego rozwiązywania układów równań liniowych.
  
  \item \textbf{Rozkład QR:} Faktoryzacja LU jest używana w algorytmach rozkładu QR, stosowanych m.in. w analizie regresji.
  
  \item \textbf{Rozwiązywanie macierzy odwrotnej:} W przypadkach, gdy macierz $A$ jest odwracalna, faktoryzacja LU może być używana do efektywnego obliczania jej odwrotności.
\end{itemize}

\section{Podsumowanie}
Faktoryzacja LU macierzy to ważne narzędzie w algebrze liniowej, znajdujące zastosowanie w różnych dziedzinach matematyki i inżynierii. W artykule tym omówiliśmy teorię faktoryzacji LU, przedstawiliśmy algorytmy obliczania, a także zastosowania praktyczne tej operacji, takie jak rozwiązywanie układów równań czy analiza regresji.

\end{document}
