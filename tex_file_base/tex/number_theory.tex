\documentclass[12pt]{article}
\usepackage{amsmath, amssymb, amsthm}

\title{Teoria Liczb: Wprowadzenie do Liczb Pierwszych}
\author{Twoje Imię i Nazwisko}
\date{\today}

\begin{document}

\maketitle

\section{Wprowadzenie}
Teoria liczb jest jednym z najstarszych działów matematyki, zajmującym się badaniem własności liczb całkowitych. W ramach tej dziedziny, liczby pierwsze odgrywają kluczową rolę. Artykuł ten ma na celu wprowadzenie czytelnika do podstawowych pojęć związanych z liczbami pierwszymi oraz przedstawienie kilku interesujących twierdzeń.

\section{Liczby Pierwsze}
Liczby pierwsze są fundamentalnymi obiektami w teorii liczb. Definiujemy je jako liczby naturalne większe od 1, które posiadają dokładnie dwa dzielniki: 1 i samą siebie. Przykłady to 2, 3, 5, 7, itd.

\section{Twierdzenie o Nieskończoności Liczb Pierwszych}
Jednym z najważniejszych twierdzeń w teorii liczb jest twierdzenie o nieskończoności liczb pierwszych. Mówi ono, że istnieje nieskończenie wiele liczb pierwszych. Dowód tego twierdzenia został przedstawiony przez Euklidesa w III wieku p.n.e. i opiera się na założeniu przez zaprzeczenie.

\begin{theorem}
Istnieje nieskończenie wiele liczb pierwszych.
\end{theorem}

\begin{proof}
Przypuśćmy, że mamy skończoną listę liczb pierwszych $p_1, p_2, \ldots, p_n$. Rozważmy liczbę $P = p_1 \cdot p_2 \cdot \ldots \cdot p_n + 1$. Zauważmy, że $P$ nie jest podzielna przez żadną z liczb pierwszych $p_1, p_2, \ldots, p_n$, ponieważ pozostawia resztę 1 po dzieleniu przez każdą z nich. Zatem $P$ musi mieć co najmniej jeden nowy dzielnik pierwszy, który nie znajduje się na naszej początkowej liście, co kończy dowód.
\end{proof}

\section{Podsumowanie}
Teoria liczb, a zwłaszcza liczby pierwsze, stanowią fascynujący obszar matematyki. W artykule tym omówiliśmy podstawowe pojęcia związane z liczbami pierwszymi oraz przedstawiliśmy kluczowe twierdzenie o nieskończoności liczb pierwszych. Oczywiście, teoria liczb to obszerna dziedzina, a ten artykuł stanowi jedynie skromne wprowadzenie.

\end{document}
