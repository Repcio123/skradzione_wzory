\documentclass[12pt]{article}
\usepackage{amsmath, amssymb, amsthm}

\title{Własności Koła: Analiza Podstawowa}
\author{Twoje Imię i Nazwisko}
\date{\today}

\begin{document}

\maketitle

\section{Wprowadzenie}
Koło to jedna z podstawowych figur geometrycznych, której własności są szeroko stosowane w matematyce, fizyce, inżynierii i wielu innych dziedzinach nauki. W tym artykule omówimy podstawowe definicje związane z kołem oraz przedstawimy kilka kluczowych własności.

\section{Definicje Podstawowe}
Koło to zbiór punktów w płaszczyźnie, które leżą w odległości równej promieniowi $r$ od ustalonego punktu nazywanego środkiem koła. Oznaczamy je jako $K(r)$, gdzie $r$ to promień koła.

\section{Obwód i Pole Koła}
\begin{theorem}
Obwód koła, o promieniu $r$, wynosi $C = 2\pi r$.
\end{theorem}

\begin{proof}
Obwód koła to suma długości wszystkich punktów leżących na obwodzie koła, co jest równoważne $2\pi r$.
\end{proof}

\begin{theorem}
Pole powierzchni koła, o promieniu $r$, wynosi $A = \pi r^2$.
\end{theorem}

\begin{proof}
Pole koła to suma powierzchni wszystkich punktów leżących wewnątrz koła, co jest równoważne $\pi r^2$.
\end{proof}

\section{Twierdzenie o Kącie na Okręgu}
\begin{theorem}
Dla każdego łuku $s$ na okręgu o promieniu $r$ i kącie $\theta$ zawartym między promieniem a łukiem, długość łuku jest równa $s = r\theta$.
\end{theorem}

\begin{proof}
Długość łuku to stosunek długości łuku do promienia, co jest równoważne $r\theta$.
\end{proof}

\section{Podsumowanie}
Własności koła są niezwykle ważne w wielu dziedzinach nauki. W artykule tym omówiliśmy podstawowe definicje oraz przedstawiliśmy trzy kluczowe własności: obwód i pole koła, oraz twierdzenie o kącie na okręgu. Te własności mają zastosowanie w szerokim spektrum dziedzin matematyki i nauki.

\end{document}
