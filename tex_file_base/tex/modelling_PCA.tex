\documentclass{article}
\usepackage{amsmath}
\usepackage{graphicx}
\usepackage{float}
\usepackage{geometry}
\usepackage{hyperref}
\usepackage{pgfplots}

\title{Mathematical Modeling in Science and Engineering}
\author{Your Name}
\date{\today}

\begin{document}

\maketitle

\begin{abstract}
Mathematical modeling is a powerful tool used in various scientific and engineering disciplines. This article provides an overview of mathematical modeling techniques, discusses their applications, and presents case studies. We explore the use of differential equations, optimization, and statistical models, along with real-world examples and practical insights.
\end{abstract}

\section{Introduction}

Mathematical modeling is an essential approach in science and engineering that involves using mathematical equations, models, and simulations to understand and analyze complex phenomena. In this article, we will explore different aspects of mathematical modeling, including its techniques, applications, and practical relevance.

\section{Mathematical Modeling Techniques}

Mathematical modeling encompasses various techniques, including differential equations, optimization, and statistical models. We will briefly introduce each of these techniques.

\subsection{Differential Equations}

Differential equations are fundamental in modeling dynamic systems. They describe how a quantity changes over time, such as in physics, chemistry, and engineering. A common example is the first-order linear ordinary differential equation:

\begin{equation}
\frac{dy}{dt} = k \cdot y,
\end{equation}

where $y(t)$ represents the quantity, $t$ is time, and $k$ is a constant.

\subsection{Optimization Models}

Optimization models aim to find the best solution among a set of feasible choices. Linear programming, a widely used optimization technique, is often used to optimize resource allocation, production processes, and transportation routes.

\begin{align}
\text{Maximize} \quad & Z = c_1x_1 + c_2x_2 \\
\text{Subject to} \quad & A_{ij}x_i \leq b_j \\
& x_i \geq 0
\end{align}

\subsection{Statistical Models}

Statistical models are used to analyze and interpret data. They include regression analysis, time series analysis, and hypothesis testing. A simple linear regression model is given by:

\begin{equation}
y = \beta_0 + \beta_1 x + \epsilon,
\end{equation}

where $y$ is the dependent variable, $x$ is the independent variable, $\beta_0$ and $\beta_1$ are regression coefficients, and $\epsilon$ is the error term.

\section{Applications of Mathematical Modeling}

Mathematical modeling is applied in a wide range of fields, including physics, engineering, biology, economics, and social sciences. Some key applications include:

\begin{itemize}
\item Physics: Modeling physical systems with differential equations, e.g., motion of planets.
\item Engineering: Optimizing resource allocation in manufacturing processes.
\item Biology: Modeling population growth and the spread of diseases.
\item Economics: Analyzing economic trends and predicting market behavior.
\item Social Sciences: Modeling the dynamics of social networks.
\end{itemize}

\section{Case Studies}

Let's explore two case studies to illustrate the power of mathematical modeling:

\subsection{Epidemiological Model}

In the context of the COVID-19 pandemic, mathematical modeling has played a critical role in predicting the spread of the virus and assessing the impact of different intervention strategies. The well-known SIR (Susceptible-Infectious-Removed) model is widely used to study disease dynamics.

\begin{equation}
\frac{dS}{dt} = -\beta \cdot S \cdot I, \quad \frac{dI}{dt} = \beta \cdot S \cdot I - \gamma \cdot I, \quad \frac{dR}{dt} = \gamma \cdot I,
\end{equation}

where $S$ is the number of susceptible individuals, $I$ is the number of infectious individuals, and $R$ is the number of removed (recovered or deceased) individuals.

\subsection{Transportation Optimization}

In transportation logistics, mathematical models are used to optimize delivery routes, reduce costs, and improve efficiency. Consider a case where a delivery company needs to determine the most cost-effective routes for a fleet of vehicles to deliver packages to multiple locations.

\begin{table}[H]
\centering
\begin{tabular}{|c|c|c|c|}
\hline
\textbf{Location} & \textbf{Demand (Units)} & \textbf{Distance (miles)} & \textbf{Cost (\$)} \\ \hline
A & 10 & 5 & 15 \\ \hline
B & 15 & 10 & 20 \\ \hline
C & 8 & 7 & 10 \\ \hline
D & 12 & 8 & 12 \\ \hline
\end{tabular}
\caption{Delivery Locations and Costs}
\end{table}

The optimization model would aim to minimize the total cost while satisfying the demand at each location.

\section{Conclusion}

Mathematical modeling is a versatile and powerful tool that is widely used in science and engineering. It provides insights into complex systems, facilitates decision-making, and helps solve practical problems. Understanding and applying mathematical modeling techniques are crucial for addressing real-world challenges and advancing knowledge in various disciplines.
\section{Introduction2}
\paragraph{Principal component analysis (PCA) simplifies the complexity in high-dimensional data while retaining trends and patterns. It does this by transforming the data into fewer dimensions, which act as summaries of features. High-dimensional data are very common in biology and arise when multiple features, such as expression of many genes, are measured for each sample. This type of data presents several challenges that PCA mitigates: computational expense and an increased error rate due to multiple test correction when testing each feature for association with an outcome. PCA is an unsupervised learning method and is similar to clustering1—it finds patterns without reference to prior knowledge about whether the samples come from different treatment groups or have phenotypic differences.}

\begin{figure}[h]
\centering
    $z y ^ { 2 } = 4 x ^ { 3 } - g _ { 2 } z ^ { 2 } x - g _ { 3 } z ^ { 3 }$
\end{figure}
\paragraph{PCA stuff}
\begin{figure}[h]
\centering
    $_ { g f } - i \varepsilon \frac { 1 } { 2 } \int A ^ { 2 } d ^ { 4 } x$
\end{figure}
\end{document}
