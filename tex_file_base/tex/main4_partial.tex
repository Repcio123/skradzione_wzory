\documentclass{article}
\usepackage{amsmath}

\title{The Mathematics of Flying Through a Narrow Passage}
\author{Your Name}
\date{\today}

\begin{document}

\maketitle

\section{Introduction}

Flying an aircraft through a narrow passage or between two large poles is an extreme and dangerous maneuver that should never be attempted in practice. However, understanding the mathematics and physics behind such a scenario can be an intriguing exercise in aerodynamics. In this article, we will explore the mathematical principles involved in flying through a confined space.

\section{Maneuvering Through a Narrow Passage}

To fly an aircraft through a narrow passage, pilots need to consider various factors, including the width of the passage ($w$), the speed of the aircraft ($V$), and the dimensions of the aircraft itself. One key parameter is the minimum turning radius ($R$) required to safely navigate the passage.

The mathematical relationship between the minimum turning radius, speed, and bank angle ($\theta$) can be described by the following equation:

\begin{equation}
R = \frac{V^2}{g \cdot \tan(\theta)},
\end{equation}

where:
$R$ is the minimum turning radius,
$V$ is the velocity of the aircraft,
$g$ is the acceleration due to gravity,
$\theta$ is the bank angle.

Pilots must carefully manage the bank angle to ensure the aircraft can make the turn without striking the obstacles. A more significant bank angle requires a larger turning radius, while a smaller turning radius necessitates a steeper bank angle.

\end{document}