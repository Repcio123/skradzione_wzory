\documentclass[12pt]{article}
\usepackage{amsmath, amssymb, amsthm}

\title{Trójkąt Prostokątny: Własności Boków i Zastosowania}
\author{Twoje Imię i Nazwisko}
\date{\today}

\begin{document}

\maketitle

\section{Wprowadzenie}
Trójkąt prostokątny to trójkąt, w którym jedno z kątów jest prosty (równy $90^\circ$). W tym artykule omówimy własności boków trójkąta prostokątnego oraz zastosowania tych własności w geometrii i fizyce.

\section{Twierdzenie Pitagorasa}
\begin{theorem}
W trójkącie prostokątnym, kwadrat długości przeciwprostokątnej (naprzeciwko kąta prostego) jest równy sumie kwadratów długości przyprostokątnych.
\end{theorem}

\section{Własności Boków Trójkąta Prostokątnego}
\begin{itemize}
  \item \textbf{Przyprostokątne:} Oznaczmy długości przyprostokątnych jako $a$ i $b$. Wówczas mamy:
  \[
  a^2 + b^2 = c^2
  \]
  gdzie $c$ to długość przeciwprostokątnej.
  
  \item \textbf{Wzory na Obwód:} Obwód trójkąta prostokątnego wynosi $a + b + c$.
  
  \item \textbf{Wzory na Pole:} Pole trójkąta prostokątnego wynosi $\frac{1}{2}ab$.
\end{itemize}

\section{Zastosowania Własności Trójkąta Prostokątnego}
\begin{itemize}
  \item \textbf{Geometria:} Twierdzenie Pitagorasa jest szeroko stosowane do obliczeń długości w trójkątach prostokątnych.
  
  \item \textbf{Inżynieria:} W konstrukcjach i projektowaniu, własności trójkąta prostokątnego są używane do obliczeń odległości i kątów.
  
  \item \textbf{Fizyka:} W fizyce, twierdzenie Pitagorasa jest stosowane do analizy ruchu i trajektorii.
\end{itemize}

\section{Podsumowanie}
Trójkąt prostokątny to fundamentalna figura geometryczna, a jego własności mają szerokie zastosowania w matematyce, fizyce i inżynierii. W artykule tym omówiliśmy twierdzenie Pitagorasa, wzory na długości boków, obwód i pole trójkąta prostokątnego oraz praktyczne zastosowania tych własności.

\end{document}
